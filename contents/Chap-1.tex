\chapter{Introduction: }
\label{chp:DEM}


%%%%%%%%%%%%%%%%%%%%%%%%%%%%%%%%%%%%%%%%%%%%%%%%%%%%%%%%%%%%%%%%%%%%%%%
Mozambique’s peace process in the 1990s has been widely celebrated as one of the most prominent cases of a successful war-to-peace transition in Africa, and subsequently informed both peacebuilding theory and practice (Edis 1994; Bartoli, Bui-Wrzosinka and Novak 2010). However, more recent ethnographic and historical studies suggest that Mozambique’s peacebuilding project was insufficient in creating a context for stability, growth and development free from the risk of collective violence or war (ref). Mozambique’s recent history presents a telling case of post-conflict state building and the difficulties of securing sustainable and positive peace after war. Throughout the 20th century, Mozambique found itself caught up in a global and regional ideological struggle. A country with a wealth of natural beauty and resources, Mozambique's seemingly prosperous future was greatly challenged by a long and difficult colonial past, protracted armed conflict and a flawed democratic transition. 

The decision of Renamo to return to armed conflict in 2013, two decades after signing a peace agreement with the ruling party, Frelimo, suggests that its post-conflict peacebuilding and disarmamenet, demobilisation and reintegration (DDR) project was not as successful as initially proclaimed. Since the UN Operation in Mozambique ended in 1994, internal insecurity, high rates of poverty and lack of development has persisted, manifesting as organised urban criminality and most recently a jihadi insurgency (Vines 1998; Faria 2021). 


