\chapter{Literature review and theoretical framework}
\label{chp:Lit}

State-based armed conflict can be defined as a situation where incompatibilities exist over government and/or territory where there is the use of armed force and where at least one actor is the state. Since the end of World War II, the frequency of state-based conflicts has steadily increased (Council on Foreign Relations, 2022). According to the Uppsala Conflict Data Program (UCDP), the majority of state-based conflicts on record happened during the 1990s. State-based conflicts or civil wars (internationalised or not) remain the dominant and most deadly form of conflict today. Since the 1990s, scholarly research on civil wars has developed several subfields.

The aim of this chapter is to lay the conceptual and theoretical framework for analysing the recurrence of armed conflict between Renamo and the Frelimo-led government in Mozambique two decades after the civil war. The chapter aims to assess the theoretical basis for Mozambique's peacebuilding project in 1992 whilst exploring what is understood as necessary conditions for sustainable peace within civil conflict literature. Moreover, the chapter will explore practical aspects within peacebuilding, such as DDR, in order to understand how UN missions hoped to secure \emph{sustainable} peace during the 1990s. 

The following sections provide a critical overview of the various theoretical paradigms a pplied to the study of civil conflict in relation to the question posed for this study. First, I examine the central debates pertaining to the most prominent area of study, civil war onset and then I delve into studies that have examined war duration and peacebuilding. 

\section{Agenda for (liberal) peace}

On the 75th anniversary of the United Nations, 
Former Secretary-General Boutros Boutros-Ghali's ``An Agenda for Peace'' speech










\section{DDR as a mechanism for peace}





\section{Divided - conflict recurrence and peacebuilding literature}





\section{Critical perspectives for assessing peacebuilding}





\section{Conclusion}