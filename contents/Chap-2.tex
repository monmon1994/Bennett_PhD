\chapter{Literature review and theoretical framework}
\label{chp:Lit}

The origins of modern day peacebuilding operations began at the momentous closing stages of the Cold War period in 1989. The United Nations (UN) launched a number of \emph{multidimensional} missions aimed at resolving protracted civil conflicts, which were made worse by the ideological struggle between the United States and former Soviet Union. These renewed peacebuilding missions would now not only provide technical assistance, but promote a ``liberal'' model of political and economic systems in the hopes of preventing the recurrence of future civil conflict \citep[chap.~2]{paris2004war}. Previous UN missions had focused on stabilisation and the maintaining of ceasefire agreements defined as \emph{peacekeeping}. Post-cold war the UN were finally tasked to address the pervasive and persistent violent political landscape that characterised countries across Africa, the Middle East and Central Asia throughout the 1990s. By tracing the recent evolution of peacebuilding and the critical discourse on its progress since inception, this chapter will unpack the historical and intellectual context of the theory and practice of modern peacebuilding. 

The chapter begins by examining the founding principles of peacebuilding as presented by the UN, before unpacking the logic behind DDR as a mechanism for securing peace long-term. Then explores the competing intellectual findings related to conflict recurrence and the mandate's of liberal peacebuilding experienced post-cold war. Finally, the chapter will use critical perspectives on peacebuilding to explore whether they are better suited to understand Mozambique's peace project.  

\section{Tracing the agenda for liberal peace}

In 2021, during the 75th anniversary of the UN, the \emph{New Agenda for Peace} was announced as part of Secretary-General Ant\'{o}nio Guterres's launch of the \emph{Our Common Agenda} report. The report's re-emphasise on collective peace and security is justified as a response to the current and emerging risks and trends that international conflict management is facing. The report's announcement comes 30 years after the notable \emph{An Agenda for Peace} report by Former UN Secretary-General Boutros Boutros-Ghali in 1992. Boutros-Ghali's report introduced post-conflict peacebuilding as ``action to identify and support structures which will tend to strengthen and solidify peace in order to avoid a relapse into conflict'' (ref report with paragraph).  The re-emphasise on an ``agenda for peace'' in Guterres's recent report suggests that securing peace remains a significant challenge for the UN and the international community. 

Boutros-Ghali's new vision for peace operations introduced an array of taxonomy that would characterise and differentiate between various missions. Peacekeeping, peace enforcement, and post-accord peacebuilding were carefully defined as separate operations. Unlike peacekeeping, which focuses on monitoring of ceasefire agreements, peace enforcement operations would facilitate the use of coercive measures. These measures included the use of military force for purposes other than self-defence in order to respond to ``outright aggression, imminent or actual'' \citep[44]{ghali1992}. Post-conflict peacebuilding was the third category Boutros-Ghali introduced in his report. He detailed how this element of a peace mission would play a central role in establishing structures that would help consolidate peace and avoid a renewal of future conflict. The report specifically referred to a number of activities relating to disarmament and security sector advisory but most notably encourages the advancement of liberal democratic norms. These activities have become central themes across peacebuilding literature, namely; security, political institutions, economic progress, justice and reconciliation. 

- continue to unpack each aspect of the liberal model by explaining what is understood in the literature about the importance of these activities in the UN's peacebuilding model. 
- What are the theoretical justifications for these activities?





\section{DDR as a mechanism for peace}





\section{Divided - conflict recurrence and peacebuilding literature}





\section{Critical perspectives for assessing peacebuilding in Mozambique}





\section{Conclusion}